\documentclass[]{article}
\usepackage[utf8]{inputenc}


\title{Resumen de contenidos c2}
\author{}
\date{}
\usepackage[spanish]{babel}
\usepackage{amsmath,amsfonts,amssymb,amsthm}
\usepackage{natbib}
\usepackage{graphicx,float}
\usepackage[margin=1.0in]{geometry}

%\newcommand{comando kliao}{comando bueno}
%\DeclareMathOperator\cis{cis}
%\DeclareMathOperator\Arg{Arg}

\begin{document}

\maketitle


\part{Conceptos}



\section{Apunte 6}

\begin{enumerate}
	\item Definición de matriz
	\item conjuntos de matrices
	\item Algunas definiciones
		\begin{enumerate}
			\item matriz cuadrada
			\item matriz fila y matriz columna
			\item matriz nula
		\end{enumerate}
	\item condiciones para igualdad de matrices
	\item operaciones entre matrices (para las propiedades considerar fuertemente la \textbf{cantidad de filas y columnas} de cada matriz)
		\begin{enumerate}
			\item Producto de una matriz con una constante o escalar
			\item suma de matrices y sus propiedades (conmutatividad, asociatividad elemento neutro, inverso aditivo, etc).
		\end{enumerate}
	\item producto de matrices y sus propiedades.
	\item matriz identidad
	\item matriz invertible (que tienen inverso multiplicativo)
	\item matriz inversa
	\item matriz inertible, matriz no singular
	\item determinante
	
\end{enumerate}



\clearpage





\part{Resumen}



\section{Propiedades de matrices}

\subsection{Suma}
\begin{enumerate}
	\item conmutatividad
	\item Asociatividad. $A+(B+C)=(A+B)+C$
	\item neutro para la suma es $\theta$. $A+ \theta =A$
	\item $A+(-A)= \theta$
\end{enumerate}



\subsection{Producto matriz por escalar}
\begin{enumerate}
	\item $0A=\theta$
	\item $\alpha (\beta A) = (\alpha \beta)A$
	\item $(\alpha + \beta)A= \alpha A + \beta A$
	\item $\alpha (A+B)= \alpha A + \alpha B$
\end{enumerate}



\subsection{Producto entre matrices}
\begin{enumerate}
	\item \textbf{NO} conmutativa
	\item Es asociativa. $A(BC)=(AB)C$
	\item $A\theta = \theta$
	\item $AI=A$
	\item \textbf{NO es cierto} que $AB=\theta \Longrightarrow A=\theta \vee B=\theta$
	\item no siempre existe inverso multiplicativo
	\item $A(B+C) = AB+AC \wedge  (A+B)C=AC+BC$
\end{enumerate}



\subsection{Inverso multiplicativo}
\begin{enumerate}
	\item Si $A$ tiene inversa, es \textbf{única}
	\item si $A$ es invertible, $A^{-1}$ también y $(A^{-1})^{-1}=A$
	\item si $A$ y $B$ son invertibles, $AB^{-1}$ también y $(AB)^{-1}=B^{-1}A^{-1}$
	\item Si $A$ es invertible $A^{T}$ tambien lo es y $(A^{T})^{-1}=(A^{-1})^{T}$
	\item $A \cdot A^{-1}=I$
\end{enumerate}



\subsection{Matriz transpuesta}
\begin{enumerate}
	\item $(A^{T})^{T}=A$
	\item $(A+B)^{T}=A^{T}+B^{T}$
	\item $(\lambda A)^{T}= \lambda (A^{T})$
	\item Sean $A$ y $B$ matrices multiplicables, entonces $(AB)^{T}=B^{T}A^{T}$
\end{enumerate}



\subsection{Determinante}
\begin{enumerate}
	\item $det(A)=det(A^{T})$
	\item Si todos los elementos en una fila o columna de $A$ son ceros, entonces $det(A)=0$
	\item $det(AB)=det(A) \cdot det(B)$
	\item Si $A \in M_{n}(\mathbb{K})$ y $\lambda \in (\mathbb{K})$, entonces $det(\lambda A)=\lambda^{n}det(A)$
\end{enumerate}






\end{document}