\documentclass[]{article}
\usepackage[utf8]{inputenc}


\title{Resumen química I c2}
\author{}
\date{}
\usepackage[spanish]{babel}
\usepackage{amsmath,amsfonts,amssymb,amsthm}
\usepackage{natbib}
\usepackage{graphicx}
\usepackage[margin=1.0in]{geometry}

%\newcommand{comando kliao}{comando bueno}
%\DeclareMathOperator\cis{cis}
%\DeclareMathOperator\Arg{Arg}

\begin{document}



\end{document}


\begin{document}
\maketitle

\section{El estado sólido}



\subsection{Sólidos cristalinos}
Un \textbf{sólido cristalino} posee un ordenamiento estricto y regular, es decir, sus átomos, moléculas o iones ocupan posiciones específicas. Las fuerzas que mantienen la estabilidad de un cristal pueden ser iónicas, covalentes, de van der waaals, puntes de hidrogeno o una combinación de ellas.

\begin{itemize}
	\item La distribución de las particulas en solidos cristalinos maximisa las fuerzas netas de atracción molecular.
	\item Los sólidos cristalinos suelen tener superficies planas que forman ángulos entre si.
	\item Ejemplos: Cloruro de sodio, cristal de cuarzo, diamante.
\end{itemize}

Una celda unitaria es la unidad básica que se repite en un sólido cristalino; tiene ángulos y aristas. Existen varios tipos de celdas unitarias:

\begin{enumerate}
	\item cúbica (fluorita)
	\item tetragonal (calcopirita)
	\item ortorrómbica (aragonita)
	\item romboédrica (calcita)
	\item hexagonal (esmeralda)
	\item monoclínica (azurita)
	\item triclínica (rodonita)
\end{enumerate}

Hay tres tipos de celdas cúbicas: cúbica simple, cúbica centrada en el cuerpo, cúbica centrada en las caras (la más compacta).



\subsection{Tipos de cristales}



\subsubsection{Cristales iónicos}
Formado por iones (ejemplo Cloruro de sodio). Unidos por atracción \textbf{electroestática}.

\begin{itemize}
	\item Formados for especies cargadas
	\item Aniones y cationes suelen ser de distinto tamaño
	\item Se mantienen unidos por enlaces iónicos
	\item Suelen tener \textbf{puntos de ebullición elevados}
	\item \textbf{No conducen electricidad}. En estado fundido o disueltos en agua conduce electricidad (electrolito)
	\item Son duros (resistentes al rayado) y frágiles (se rompen facilmente)
\end{itemize}



\subsubsection{Cristales covalentes}
Átomos unidos en grandes redes por \textbf{enlaces covalentes} (ejemplo: los dos alótropos del carbono, diamante y grafito).

Estructura cristalina del diamante:

\begin{itemize}
	\item cada átomo está enlazado de manera tetraédrica a otros 4 átomos de carbono
	\item los enlaces covalentes fuertez le otorgan gran resistencia y un elevado punto de fusión
\end{itemize}
Estructura cristalina del grafito:
\begin{itemize}
	\item átomos se distribuyen en forma de anillos hexagonales, cada uno enlazado a otros 3.
	\item se mantienen unidos por fuerzas débiles de van der waals por lo que sep ueden deslizar entre sí.
\end{itemize}

Los cristales covalentes poseen las siguientes caracteristicas.
\begin{itemize}
	\item son duros
	\item malos conductores eléctricos
	\item insolubles en todos los disolventes comuntes
	\item puntos de fusión muy altos
\end{itemize}



\subsubsection{Sólidos moleculares}
Unidos por fuerzas \textbf{dipolo-dipolo}, fuerzas de \textbf{disperción} o \textbf{enlaces de hidrógeno}.

\begin{itemize}
	\item son mas suaves y quebradizos que los cristales iónicos o covalentes
	\item puntos de fusión relativamente bajos (casi siempre <200°C)
	\item sunstanticas que son gases y líquidos a temperatura ambiente pueden formar sólidos moleculares a temperaturas bajas.
\end{itemize}



\subsubsection{Cristales metálicos}
Formado por metales, su empaquetamiento suele ser muy denso (modelo de mar de electrones; \textbf{deslocalización electrónica}). Son resistentes y buenos conductores de calor y electricidad.










\end{document}




